\subsection{Código MPPT}
\begin{listing}[H]
\begin{minted}{c}
float med_volt(float value){
    float prom = 0;
    for (int i = 0; i < 10; i++){
        float volt = BATTERY_ADC_RATIO * adc_read() * 3.3 / (1 << 12);
        prom = prom + volt;
        }
    return prom/10;
}    
\end{minted}
\caption{Como el código mide el voltaje en el ADC}
\label{Listing 1}
\end{listing}

\begin{listing}[H]
\begin{minted}{c}
    adc_select_input(ADC_CHANNEL_BATTERY);
    battery_voltage = med_volt(x);
    sprintf(str, "%.2f", battery_voltage);
    lcd_set_cursor(2, 4);
    lcd_string(str);
\end{minted}
\caption{Lectura ADC}
\label{Listing 2}
\end{listing}


\begin{listing}[H]
\begin{minted}{c}
if (charging_mode == BULK_MODE){
    if (battery_voltage > BULK_MAX_BATTERY_VOLTAGE) {
        // Cambio modo de carga en caso de que corriente baje del umbral
        charging_mode = ABSORTION_MODE;
    }
    else {
        lcd_set_cursor(1, 11);
        lcd_string("  BULK   ");
    }
}
\end{minted}
\caption{Lógica de cambio BULK}
\label{Listing 3}
\end{listing}

\begin{listing}[H]
\begin{minted}{c}
if (charging_mode == BULK_MODE) {
    float error = (BULK_MAX_CURRENT_VOLTAGE - battery_current) * 50;
    if (battery_current > BULK_MAX_CURRENT_VOLTAGE) {
        pwm_level = pwm_level - (-1 * error * INTEGRAL_CONSTANT);
    }
    else {
        pwm_level += 1 * error * INTEGRAL_CONSTANT;
    }
    // Verifico que no exceda los limites
    pwm_level = saturador(PWM_WRAP, pwm_level);
    // Ajusto PWM
    pwm_set_gpio_level(pwm, pwm_level);
}
\end{minted}
\caption{Lógica del PWM en modo BULK}
\label{Listing 4}
\end{listing}

\begin{listing}[H]
\begin{minted}{c}
inline static uint16_t saturador(uint16_t wrap, int16_t level) {
    if(level > PWM_WRAP) {
        level = PWM_WRAP;
        return PWM_WRAP;
    }
    else if(level < 0) {
        level = 0;
        return 0;
    }
    return level;
}
\end{minted}
\caption{Saturador}
\label{Listing 5}
\end{listing}

\begin{listing}[H]
\begin{minted}{c}
    lcd_set_cursor(3, 14);
    sprintf(str, "%d", (int) (pwm_level * 0.0264061262));
    lcd_string(str);
\end{minted}
\caption{PWM a LCD}
\label{Listing 6}
\end{listing}

\begin{listing}[H]
\begin{minted}{c}
int64_t alarm_callback(alarm_id_t id, void *user_data) {
    if(cont_minute < 60){
        prom_minute += (battery_in * battery_current);
        cont_minute += 1;
    }
    else{
        prom_hour += (prom_minute / cont_minute);
        cont_hour += 1;
        cont_minute = 0;
    }
    return 0;
}
add_alarm_in_ms(-1000, alarm_callback, NULL, false);
\end{minted}
\caption{Interrupción cada 1 segundo}
\label{Listing 7}
\end{listing}

\begin{listing}[H]
\begin{minted}{c}
void on_uart_rx() {
    while (uart_is_readable(UART_ID)) {
        if (uart_is_writable(UART_ID)) {
            real_prom = prom_hour / cont_hour;
            sprintf(json, "{\"prom_hour\":%f,\"volt_actual\":%f}",
            real_prom, battery_voltage);
            uart_putc(UART_ID, *json);
        }
    }
}
\end{minted}
\caption{Interrupción UART}
\label{Listing 8}
\end{listing}

\subsection{Código ESP32}

\begin{listing}[H]
\begin{minted}{python}

class Solarlink(object):

    def __init__(self):

        # direccion i2c ADC externo
        self.i2c_corriente_dir = 72
        # scl
        self.scl = 22
        # sda
        self.sda = 21
        # freq i2c
        self.freq = 400000
        # gain sensor
        self.gain = 0
        # pendiente de sensor de corriente
        self.pendiente_corriente = 10.1288571642
        # pin adc sensor voltaje
        self.adc_pin = 33
        # atenuacion ADC
        self.atten = ADC.ATTN_2_5DB
        # referencia sensor voltaje
        self.ref = 221 / 0.5150402

\end{minted}
\caption{Main parte 1}
\label{main-1}
\end{listing}


\begin{listing}[H]
\begin{minted}{python}
        # i2c init
        self.i2c = I2C(1, scl=Pin(self.scl), 
                          sda=Pin(self.sda), 
                          freq=self.freq)
        # ADC externo init
        self.ads1115 = ads1115.ADS1115(self.i2c,
                                       self.i2c_corriente_dir,
                                       self.gain)
        # ADC en el pin init
        self.adc = ADC(Pin(self.adc_pin, Pin.IN))
        # atenuacion ADC
        self.adc.atten(self.atten)
        # display init
        self.lcd = I2cLcd(self.i2c, 0x27, 4, 20)

        # timer 0
        self.timer0 = None
        # trigger fin mediciones
        self.fin_mediciones = False

        # medicion final
        self.medicion = None
}
\end{minted}
\caption{Main parte 2}
\label{main-2}
\end{listing}

\begin{listing}[H]
\begin{minted}{python}
def callback_fin_mediciones(self, t):
        self.fin_mediciones = True
\end{minted}
\caption{Clase callback\_fin\_mediciones() la librería SolarLink}
\label{callback-mediciones}
\end{listing}

\begin{listing}[H]
\begin{minted}{python}
    # medicion de sensor de corriente en el instante
    def corriente_dif_read(self, pin_a, pin_b):
        # medicion sensor de corriente con ADC externo
        medicion = self.ads1115.raw_to_v(self.ads1115.read(7, pin_a, pin_b))
        # traspaso a RMS
        corriente = self.pendiente_corriente * medicion / math.sqrt(2)

        return corriente
\end{minted}
\caption{Clase corriente\_dif\_read() de la librería SolarLink}
\label{clase corriente}
\end{listing}

\begin{listing}[H]
\begin{minted}{python}
    # medicion de sensor de voltaje en el instante
    def voltaje_read(self):
        # mido sensor de voltaje en el pin ADC, valor en RMS
        voltaje = self.adc.read_uv() / 1000000 * self.ref

        return voltaje
\end{minted}
\caption{Clase voltaje\_read() de la librería SolarLink}
\label{clase voltaje}
\end{listing}

\begin{listing}[H]
\begin{minted}{python}
       def medicion_default_segundo(self):
        # timer 0 init, timer de fin de mediciones
        self.timer0 = Timer(0).init(period=1000, mode=Timer.ONE_SHOT, 
                                    callback=self.callback_fin_mediciones) 


        # pico de corriente en ambas lineas (l1, l2) en el tiempo de medicion
        pico_corriente_l2 = 0
        suma_voltaje = 0 # suma de valores de voltaje para promediar
        index = 0         # index para promediar
       
        while True:
            suma_voltaje += self.voltaje_read()
            index += 1
            # mido sensores de corriente
            corriente_actual_l2 = self.corriente_dif_read(0, 1)

            # si la corriente medida es mas alta que el pico previo, sobreescribo
            if corriente_actual_l2 > pico_corriente_l2:
                pico_corriente_l2 = corriente_actual_l2
                
            # si corta el timer
            if self.fin_mediciones:
                # saco promedio de voltaje
                voltaje = suma_voltaje / index
                # saco corriente linea 1 y 2
                corriente_l2 = pico_corriente_l2

                # calculo consumo linea 1 y 2
                consumo_l2 = voltaje * corriente_l2
               
                # guardo en el atributo las mediciones
                self.medicion = {"voltaje": int(voltaje),
                                "corriente_l2": round(corriente_l2, 2),
                                "consumo_l2": int(consumo_l2)}
                # reinicio variables                
                pico_corriente_l2 = 0
                suma_voltaje = 0
                index = 0
                # reinicio la variable de callback
                self.fin_mediciones = False
                # rompo el bucle y retorno
                return self.medicion
\end{minted}
\caption{Clase medicion\_default\_segundo(), ejemplo midiendo linea 2}
\label{medicion default segundo}
\end{listing}

\begin{listing}[H]
\begin{minted}{python}
while 1:
    valores = solarlink.medicion_default_segundo()

    voltaje = valores["voltaje"]
    corriente_l1 = valores["corriente_l1"]
    corriente_l2 = valores["corriente_l2"]
    consumo_l1 = valores["consumo_l1"]
    consumo_l2 = valores["consumo_l2"]

    p_l1.off()
    p_l2.off()
   
    trigger = 100

    if consumo_l1 + consumo_l2 < trigger:
        p_l1.on()
        p_l2.on()
    else:
        if consumo_l1 < trigger and consumo_l2 < trigger:
            if consumo_l1 > consumo_l2:
                p_l1.on()
                p_l2.off()
            else:
                p_l1.off()
                p_l2.on()
        if consumo_l1 > trigger and consumo_l2 > trigger:
            p_l1.off()
            p_l2.off()
        elif consumo_l1 > trigger and consumo_l2 < trigger:
            p_l1.off()
            p_l2.on()
        elif consumo_l1 < trigger and consumo_l2 > trigger:
            p_l1.on()
            p_l2.off()
\end{minted}
\caption{Bucle principal del main.py}
\label{bucle main}
\end{listing}

\subsection{Código web}


\begin{listing}[H]
\begin{minted}{python}
urlpatterns = [
    path('admin/', admin.site.urls),
    # path de la app home
    path('', include("home.urls")),
    # path de la app user_mngmnt
    path('user/', include('user_mngmnt.urls'))
]
\end{minted}
\caption{urls.py raíz}
\label{urls.py_raiz}
\end{listing}

\begin{listing}[H]
\begin{minted}{python}
# páginas host permitidas
ALLOWED_HOSTS = ['.vercel.app', 'solarlink.ar']
# Application definition
INSTALLED_APPS = [
    # apps
    'home',
    'user_mngmnt',
    # modulo django-crontab
    "django_crontab",
    # modulo django-compressor
    "compressor"]
MIDDLEWARE = [
    # middleware de white noise
    'whitenoise.middleware.WhiteNoiseMiddleware']
    
# Static files (CSS, JavaScript, Images)
STATIC_ROOT = "static/"
STATIC_URL = "static/"

# django-compress configuracion
COMPRESS_PRECOMPILERS = (
    ('text/x-scss', 'django_libsass.SassCompiler'),)
    
# buscadores de staticfiles
STATICFILES_FINDERS = (
    # django-compressor
    'compressor.finders.CompressorFinder',)
# default URL para redireccionar cuando no está logueado
LOGIN_URL = "login"

# CRONTAB
CRONJOBS = [
    # todos los dias a 00:45
    ('45 0 * * *', 'user_mngmnt.cron.ordenador'),
    # cada 10 mins
    ('*/10 * * * *', 'user_mngmnt.cron.token_clean')]
    
# EMAIL
EMAIL_HOST = 'smtppro.zoho.com'
EMAIL_HOST_USER = 'no-reply@solarlink.ar'
# nombre custom
DEFAULT_FROM_EMAIL= 'Solar Link Accounts<no-reply@solarlink.ar>'
\end{minted}
\caption{Configuraciones especiales para el proyecto en el settings.py}
\label{settings.py}
\end{listing}

\begin{listing}[H]
\begin{minted}{django}

- HEAD, cargo los static y el compresor -



<!DOCTYPE html>
<html lang="es">

<head>
<title>Home</title>
<link rel="stylesheet" href="">

    Compresor para poder usar static directions en el master.scss


<link rel="stylesheet" href="" />



</head>

-BARRA SUPERIOR- Si el usuario está logueado


<li class="nav__items">
    <p class="nav__links"> ¡Hola, {{ request.user.first_name }}!</p>
</li>
<li class="nav__items">
    <a href=""class="nav__links">Mis datos</a>
</li>
<li class="nav__items">
    <a href=""class="nav__links">Cerrar sesión</a>

<li class="nav__items">
    <a href=""class="nav__links">Iniciar sesión</a>
</li>
<li class="nav__items">
    <a href=""class="nav__links">Registrarse</a>
</li>

 Aqui se extiende el header de home.html 


\end{minted}
\caption{Secciones head y barra superior del home.html}
\label{home.html}
\end{listing}

\begin{listing}[H]
\begin{minted}{python}
urlpatterns = [
    # path home
    path("", views.index, name="index"),
    # path galeria
    path("galeria", views.galeria, name="galeria")
]
\end{minted}
\caption{urls.py de la app home}
\label{urls.py_home}
\end{listing}

\begin{listing}[H]
\begin{minted}{python}
# vista de home
def index(request):
    return render(request, "home/index.html")
# vista de galeria
def galeria(request):
    return render(request, "home/galeria.html")
\end{minted}
\caption{views.py de la app home}
\label{views.py_home}
\end{listing}


\begin{listing}[H]
\begin{minted}{python}
class Signup(View):
    # si se rellena un login
    def post(self, request):     
        # consigo el form con los datos posteados
        form = SignupForm(request.POST)

        # si el formulario es valido
        if form.is_valid():
            # tomo los datos
            first_name = form.cleaned_data['first_name']
            last_name = form.cleaned_data['last_name']
            email = form.cleaned_data['email']
            username = form.cleaned_data['username']
            password1 = form.cleaned_data['password1']
            password2 = form.cleaned_data['password2']

            # creo el usuario
            User.objects.create_user(email=email, 
                                        username=username, 
                                        password=password1, 
                                        first_name= first_name, 
                                        last_name = last_name)
            # lo autentico y logueo
            user = auth.authenticate(username=username, password = password1)
            auth.login(request, user)
            # genero token de confirmacion de registro
            token = secrets.token_urlsafe(32)
            # lo guardo en la base de datos
            models.UsersTokens(user=user, signup_token=token).save()
            # deshabilito al usuario hasta que verifique por mail
            user.is_active = False
            user.save()
            
            # mando mail
            no_reply_sender.delay(#argumentos y ctx)

            # redirijo a pestaña a continuación
            return render(request, 'user_mngmnt/auth/signup_verification.html')    
    # si entro a la web con un GET
    def get(self, request, *args, **kwargs):
        form = SignupForm()
        return render(request, 'user_mngmnt/auth/signup.html', {'form': form})
\end{minted}
\caption{Lógica del registro}
\label{signup_user_mngmnt}
\end{listing}

\begin{listing}[H]
\begin{minted}{python}

# verificacion de registro
class SignupVerification(View):
    def get(self, request, token):
        # busco en la tabla de tokens un token igual al ingresado
        data = models.UsersTokens.objects.filter(signup_token=token)
        # si el token esta entre los tokens guardados
        if data and token in data[0].signup_token:
            user = data[0].user
            # activo al usuario
            user.is_active = True
            user.save()
            # borro el token
            models.UsersTokens.objects.filter(signup_token = token).delete()
            # aviso por mail que el usuario se ha verificado
            no_reply_sender.delay(email=user.email, 
                                  subject="¡Tu cuenta se ha verificado!",
                                  html_message=f"""
¡Felicidades, {user.first_name}! El usuario de Solar Link 
{user.username} se ha verificado. ¡Ya podés acceder a la 
plataforma!""")
            # devuelvo que el usuario se registro adecuadamente
            return render(request, 'user_mngmnt/auth/signup_verification.html')
        # si no hay token o es invalido, devuelvo signup
        else:
            return redirect('signup')
\end{minted}
\caption{Lógica de la verificación de registro}
\label{signup_verification_user_mngmnt}
\end{listing}

\begin{listing}[H]
\begin{minted}{python}
# Password reset
class PasswordReset(View):
    def get(self, request):
        return render(request, "user_mngmnt/auth/password-reset.html")
    
    def post(self, request):
        # tomo mail
        email = request.POST['email']
        # busco usuarios con ese mail
        users = User.objects.filter(email=email)
        # si hay usuarios
        if users:
            # para cada usuario
            for user in users:
                # genero token
                token = secrets.token_urlsafe(32)
                # guardo token a usuario
                models.UsersTokens(user=user, password_rst_token=token).save()

                # mando mail
                no_reply_sender.delay(email = user.email, 
                subject='Cambio de contraseña',
                html_message=render_to_string(
                            "user_mngmnt/auth/confirmacion_password.html",
                            context))
        # devuelvo vista con booleano para avisar que ya se envió mail
        return render(request, 
                      'user_mngmnt/auth/password-reset.html', 
                      {'done': True})
\end{minted}
\caption{Lógica del password reset}
\label{password_reset_user_mngmnt}
\end{listing}

\begin{listing}[H]
\begin{minted}{python}
class Login(View):

    def get(self, request):
        form = LoginForm()
        return render(request, "user_mngmnt/auth/login.html", {"form":form})
    
    def post(self, request):
        form = LoginForm(request.POST)

        # si el form es valido
        if form.is_valid():
            #obtengo user y pass
            username = form.cleaned_data['username']
            password = form.cleaned_data['password']

            # autentico si existe usuario
            user = auth.authenticate(username=username, password = password)
            # logueo
            auth.login(request, user)
            # redirijo a index
            return redirect('index2')
        # si el form no es valido
        else:
            # codigo de error
            try:
                # intento encontrar mensaje de error
                error = form.errors.as_data()['__all__'][0].message
            # si no lo encuentro
            except:
                # hago diccionario de errores
                errors = form.errors.as_data()
                # para cada error
                for i in errors:
                    # busco el ultimo mensaje
                    error = errors[i][0].message
                # si el mensaje es el mencionado abajo
                if error == 'This field is required.':
                    # devuelvo error
                    error = 'Caracteres no validos'
                
            return render(request, 'user_mngmnt/auth/login.html', 
                          {"form": form, 'error': error})
\end{minted}
\caption{Lógica del login}
\label{login_user_mngmnt}
\end{listing}





\begin{listing}[H]
\begin{minted}{python}
# no permite acceder a las pestañas estando logueado
def unlogued_required(function, redirect_link):
    if function:
        def check(request):
            if request.user.username == '':
                return function(request)
            else:
                return redirect(redirect_link)
        return check

    else:
        def decorator(func):
            def check(request):
                if request.user.username == '':
                    return func(request)
                else:
                    return redirect(redirect_link)
            return check
        return decorator

urlpatterns = [
    # login
    path('login/', unlogued_required(views.Login.as_view(), 
         redirect_link='index'), name='login'),
    path('logout/', views.logout, name='logout'),
    # signup
    path("signup/", unlogued_required(views.Signup.as_view(), 
         redirect_link='index'), name="signup"),
    path("signup-verification/<token>/", views.SignupVerification.as_view(), 
         name="signup_verification"),
    # password reset
    path("password-set/<token>", views.PasswordSet.as_view(), 
         name="password_set"),
    path("password-set/", views.PasswordSet.as_view(), 
         name="password_set"),
    path("password-reset/", views.PasswordReset.as_view(), 
         name="password_reset"),
    # USER INFO #
    path("api-login/", csrf_exempt(views.APILogin.as_view()), 
         name="api_login"),
    path("load-data/", csrf_exempt(views.LoadData.as_view()),
         name="load_data"),
    path("index/", views.index, name="index2")]
\end{minted}
\caption{urls.py de la app User Management}
\label{urls.py_user_mngmnt}
\end{listing}


\begin{listing}[H]
\begin{minted}{python}
class LoginForm(forms.Form):
    username = forms.CharField(label='Username:', max_length=16,
               widget=forms.TextInput(attrs={'placeholder': 'Username', 
                                             "class": "controls"}))
    password = forms.CharField(label='Contraseña:', max_length=32, 
               min_length=8, 
               widget=forms.PasswordInput(attrs={'placeholder':'Contraseña', 
                                                 "class": "controls"}))

    def clean(self):
        try:
            password = self.cleaned_data["password"]
            username = self.cleaned_data["username"]
        except:
            # si hay problemas con lo ingresado, derivo la resolucion al backend
            return self.cleaned_data
        # me fijo si existe un usuario con ese username
        username_confirmation = User.objects.filter(username = username)
        # autentico si existe un usuario con ese username y contraseña
        user = auth.authenticate(username=username, password = password)

        # si no existe el usuario
        if not username_confirmation:
            raise ValidationError('El usuario no existe')

        # si existe el usuario pero su cuenta no está activa
        if username_confirmation and not username_confirmation[0].is_active:
            raise ValidationError('Verifica tu cuenta para iniciar sesión')
        
        # si existe el usuario pero no con esa contraseña
        elif username_confirmation and not user:
            raise ValidationError('La contraseña es incorrecta')
        
        return self.cleaned_data
\end{minted}
\caption{forms.py de la app User Management}
\label{forms.py_user_mngmnt}
\end{listing}


\begin{listing}[H]
\begin{minted}{python}
# para cada usuario
for user in users:
    # datos del usuario
    user_data = models.DatosHora.objects.filter(user=user)
    # mientras existan datos
    while user_data:
        # variables temporales
        voltaje_dia_red = []
        consumo_dia_red = 0
        consumo_dia_solar = 0
        solar_por_hora = []
        potencia_dia_panel = 0
        horas_de_carga = []
        voltajes_bateria = []
        errores = []

        # referencia para dia, mes
        referencia = user_data[0]
        #datos del dia buscado
        dia_data = user_data.filter(dia=referencia.dia, mes=referencia.mes, 
                                    año=referencia.año)

        # para cada dato del dia
        for data in dia_data:
            # acumulo en valores sumario diario
            voltaje_dia_red.append(data.voltaje_hora_red)
            consumo_dia_solar += data.consumo_hora_solar
            consumo_dia_red += data.consumo_hora_red

            solar_por_hora.append(data.solar_ahora)
            potencia_dia_panel += data.panel_potencia
            horas_de_carga.append(data.cargando)
            voltajes_bateria.append(data.voltaje_bateria)

            errores.append(data.errores)
            product_id = data.product_id

            # borro el dato
            data.delete()
##############################################################################
# terminada la acumulacion diaria, se guarda en otra tabla de la base de datos
# y se actualiza user_data
##############################################################################
\end{minted}
\caption{Algoritmo de limpieza y resumen de la base de datos}
\label{algoritmo_sorter_user_mngmnt}
\end{listing}



\begin{listing}[H]
\begin{minted}{python}
def token_clean():
    # todos los tokens activos
    data = models.UsersTokens.objects.all()
    # hora en timezone
    actual = timezone.now()
    # para cada dato
    for d in data:
        # si el tiempo entre que el token fue creado y el actual es mayor a 2hs
        if (actual - d.time) > datetime.timedelta(hours=2):
            # borro el token
            d.delete()
\end{minted}
\caption{Función que borra los tokens}
\label{algoritmo_borra_tokens_user_mngmnt}
\end{listing}

\begin{listing}[H]
\begin{minted}{python}
# datos del micro por hora
class DatosHora(models.Model):
    # usuario
    user = models.ForeignKey(User, on_delete=models.CASCADE, default=None)
    # promedio del voltaje en la hora proveniente de la red
    voltaje_hora_red = models.FloatField(default=None)
    # consumo total en la hora de origen solar
    consumo_hora_solar = models.FloatField(default=None)
    # consumo total en la hora proveniente de la red
    consumo_hora_red = models.FloatField(default=None)
    # consumo linea 1
    consumo_l1 = models.FloatField(default=None)
    # consumo linea 2
    consumo_l2 = models.FloatField(default=None)


    hora = models.IntegerField(default=None)
    dia = models.IntegerField(default=None)
    mes = models.IntegerField(default=None)
    año = models.IntegerField(default=None)

    # booleano que indica si las lineas estan alimentadas ahora 
    # por el sistema solar
    solar_ahora = models.BooleanField(default=None)
    # potencia entregada por el panel en esa hora
    panel_potencia = models.IntegerField(default=None)
    # booleano que indica si la bateria esta cargando ahora
    cargando = models.BooleanField(default=None)
    # voltaje de la bateria actual, con esto se puede sacar el porcentaje 
    # de la bateria
    voltaje_bateria = models.IntegerField(default=None)

    # booleano que indica que en esta hora hubo errores
    errores = models.BooleanField(default=None)

    # id de producto
    product_id = models.CharField(max_length=50)

    class Meta:
        ordering = ["user", "año", "mes", "dia", "hora"]
\end{minted}
\caption{Tabla de la base de datos de los datos por hora}
\label{models.py_DatosHora_user_mngmnt}
\end{listing}

\begin{listing}[H]
\begin{minted}{python}
@shared_task()
def no_reply_sender(email, subject, html_message):
    mail = EmailMessage(subject, html_message, to=[email])
    # aclaracion de tipo de contenido
    mail.content_subtype = 'html'
    mail.send()
\end{minted}
\caption{Subproceso que manda mails}
\label{tasks.py_mails_user_mngmnt}
\end{listing}


\begin{listing}[H]
\begin{minted}{json}
{
    "builds": [
      {
        "src": "SolarLinkWebApp/wsgi.py",
        "use": "@vercel/python"
      }
    ],
    "routes": [
      {
        "src": "/(.*)",
        "dest": "SolarLinkWebApp/wsgi.py"
      }
    ]
}
\end{minted}
\caption{vercel.json}
\label{vercel.json}
\end{listing}

\begin{listing}[H]
\begin{minted}{text}
                        amqp==5.1.1
                        asgiref==3.7.2
                        beautifulsoup4==4.12.2
                        billiard==4.1.0
                        bs4==0.0.1
                        celery==5.3.4
                        certifi==2023.7.22
                        charset-normalizer==3.2.0
                        click==8.1.7
                        click-didyoumean==0.3.0
                        click-plugins==1.1.1
                        click-repl==0.3.0
                        Django==4.2.5
                        django-appconf==1.0.5
                        django-compressor==4.4
                        django-crontab==0.7.1
                        django-libsass==0.9
                        idna==3.4
                        kombu==5.3.2
                        libsass==0.22.0
                        prompt-toolkit==3.0.39
                        python-dateutil==2.8.2
                        pytz==2023.3.post1
                        rcssmin==1.1.1
                        requests==2.31.0
                        rjsmin==1.2.1
                        six==1.16.0
                        soupsieve==2.5
                        sqlparse==0.4.4
                        tzdata==2023.3
                        urllib3==2.0.4
                        vine==5.0.0
                        wcwidth==0.2.6
                        whitenoise==6.5.0
\end{minted}
\caption{requirements.txt}
\label{requirements.txt}
\end{listing}

\clearpage

\subsection{Referencias}
\href{https://en.wikipedia.org/wiki/Maximum_power_point_tracking}{Seguidor de punto de maxima potencia, Wikipedia}\\

\href{https://en.wikipedia.org/wiki/Buck_converter}{Conversor buck, Wikipedia}\\

\href{https://es.wikipedia.org/wiki/Controlador_PID}{Controladores PID, Wikipedia}\\

\href{https://oa.upm.es/55985/1/TFG_ALBERTO_JIMENEZ_DE_LA_PENA.pdf}{Diseño de convertidor DC/DC con control reomoto para smartigrids}\\

\href{https://repositorio.uca.edu.ar/bitstream/123456789/11949/1/Bassani%20Mariano.%20TRABAJO%20FINAL%20DE%20MBA_MB_20210311_v2.pdf}{Trabajo final de Facultad de Ciencias Económicas de la UCA: ¿Es negocio invertir en generación eléctrica solar en Argentina?}\\

\href{https://www.un.org/es/chronicle/article/la-promesa-de-la-energia-solar-estrategia-energetica-para-reducir-las-emisiones-de-carbono-en-el#:~:text=De%20acuerdo%20con%20la%20Federaci%C3%B3n,ahorrar%20600%20kilogramos%20de%20CO2}{Organización de las Naciones Unidas: La promesa de la energía solar: Estrategia energética para reducir las emisiones de carbono en el siglo XXI}\\

\href{https://ri.conicet.gov.ar/handle/11336/182578?show=full}{Estado del arte de la tecnología de celdas solares basadas en semiconductores}\\

\href{https://docs.python.org/3/index.html}{Python 3.12.0 documentation}\\

\href{https://docs.djangoproject.com/en/4.2/}{Django documentation}\\

\href{https://docs.micropython.org/en/latest/}{MicroPython documentation}\\

\href{https://microdot.readthedocs.io/en/latest/}{Microdot documentation}\\

\href{https://github.com/solarlink-ar/solarlink}{Solar Link GitHub}\\

\href{https://solarlink.ar}{Web de Solar Link}\\
